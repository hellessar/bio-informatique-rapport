\section{What is a protein structure prediction? }
\subsection{What is a protein tertiary structure? } 
\emph{The term \underline{protein tertiary structure} refers to a protein's geometric shape. The tertiary structure will have a single polypeptide chain "backbone" with one or more protein secondary structures, the protein domains. Amino acid side chains may interact and bond in a number of ways. The interactions and bonds of side chains within a particular protein determine its tertiary structure. The protein tertiary structure is defined by its atomic coordinates. These coordinates may refer either to a protein domain or to the entire tertiary structure.}
\newline \newline
The scientific vulgarization is that the tertiary structure is the spacial 3D structur. Her shape can change depending on his structure the pH and the temperature. A consequent number or properties can be found thank to the tertiary structure.
\newline
Thus, we can easily conclude why it is a major importance to modelize such data.


\subsection{Protein tertiary structure determination}

A protein tertiary structure determination allows us to know more on the protein we are currently observing such as:
\begin{itemize}
	\item Connection between sequence and structures
	\item  Easier than microscopic observation
	\item Evolution of proteins
\end{itemize}

\subsection{Different protein tertiary structure determination methods} 
\subsubsection{Template-based modelling}
\subsubsubsection{Template-based modelling fold recognition}
Similar protein's sequences have the same folds. Thus, we can classify proteins according to their shapes. One of the advantages of such classification is that the number of unique structural folds is very low compares to the number of proteins.
\newline
Today, the classification of folds is very advanced and only a few folds are discovered.
\subsubsubsection{Template-based modelling homology }
Homology modelling is different and more complex than fold recognition. Instead of using the structural form of the protein the study of the amino acid sequence is used. Therefor, the complexity is bigger, the manippulation harder but the accuracy is more advance.

\subsubsection{Template-free modelling}
Template-free modeling is the prediction of the proteins structure. Compared to temple-based modelling, no proteins are used as template. 
\newline This technique has a lot of advantages compares to template-based ones. The protein will not be searched for his structural form or functions, but for some parts of his sequence. All proteins have commun function,  like energy functions or signals function for example. This technic is good because less database dependent than the templase-based one. Most functions are recorded, while functions can suffer mutation, transformation the function will not change or will be known.