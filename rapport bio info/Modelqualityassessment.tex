\section{Model quality assessment}
\subsection{Model quality asessment algorithms}
\begin{enumerate}
	\item ModFOLDclust2
		\begin{enumerate}
		\item Clustering-based method
		\item Combines structural alignement of multiple models
		\item Producing glabal quality scores and per-residue errors
		\end{enumerate}
	\item RFMQA 
		\begin{enumerate}
		\item Single model-based method
		\item Random forest based model quality assessment
		\item Ranks protein models using its structural features and knowledge-based potential energy terms
		\item Produces global model quality score
		\end{enumerate}
\end{enumerate}

Basically, 

\subsection{Structurally informed function prediction}

There are many predictions ways existing for proteins structures. Most of them are structures observation methods.
\newline
\begin{enumerate}
\item Geometric methods
\item Energetic methods
\item Homology modelling
\item Surface accessibility
\item Physiochemical properties
\end{enumerate}
Also, others methods do exist. A sequence based method exist. It has significant impact on understanding protein function, elucidating signal transduction networks. This method accentuate the study of the amino acid sequence  and his prediction.
\newline 
This method is particuliary appreciated because the number of sequences to study is cosntantly growing and the sequence study take a consequent amount of time, the prediction will save time and will be able to reveal the protein's sequence.



