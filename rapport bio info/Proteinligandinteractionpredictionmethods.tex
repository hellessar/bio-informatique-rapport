\section{Protein ligand interaction prediction methods}
\subsection{FunFOLD}
The key features to understand FunFOLD is that despite the evolution (mutation, subsitution etc...) the structures, and by extention the folds, are more well-conserved than the sequence. Plus the ligand which allowed operation such as blocking a site, activate or deactivate protein's functions etc... are also well-conserved, this imply that the expression of the proteins depends on his structure and his ligands. Thus if a protein have the same ligand and structure we can imply that they have the same function.
\newline
\subsubsection{How does it work?}
The ligand could be considered as a tiny point of glue, it is first attached on a part of the protein, then it will carry the given information.
\newline
The prediction system work like this:
\newline
A ligand glue himself on a specific part of the protein and will leave there some marks. These residues can be visible but with some imprecision. The deposit can be realesed at a variable distance. This lack of accuracy can be compensate by the structure of the protein. Indeed, there are some structure that can promote a better accuracy. So the choice of the zone to study is very improtant. 
\newline
The FUNFold technic consist on the prediction capacity to know where the ligand are going to attached themselves and then to predict the function depending on the ligand expression.
\subsubsection{Benchmarking}
By comparing the predictions values returned by the FUNFold one compare to others technics used on the CASP, we can clearly see that the mean value is higher for the FUNFold. Thus FUNFold is more efficient.
\subsubsection{FUNFOLDQA: quality assement tool}
This quality assement for FUNFold is based on protein-ligand site residue. For starter, it is needed to have a 3D structures to analyse. Then the protein structure is analysed. Many methods are used:
\begin{enumerate}
\item{BDTalign: it's basically form recognition, the closest equivalent residues is choosen from a database.}
\item{Identify score: It's the same idea as BDTalign but instead of using the 3D form the amino acid sequence is searched and then the closest equivalent is choosen according to the amino acid sequence.}
\item{Rescaled BLOSUM62, Equivalent residue ligand distance score etc...}
\end{enumerate}

Basically most methods presented could be separated in two groups: using the structure of the protein and/or ligand or the composition of the protein/ligand.