%%% LaTeX Template
%%% This template can be used for both articles and reports.
%%%
%%% Copyright: http://www.howtotex.com/
%%% Date: February 2011

%%% Preamble
\documentclass[paper=a4, fontsize=11pt]{scrartcl}	% Article class of KOMA-script with 11pt font and a4 format

\setcounter{secnumdepth}{4}
\setcounter{tocdepth}{4}
\makeatletter
\newcounter {subsubsubsection}[subsubsection]
\renewcommand\thesubsubsubsection{\thesubsubsection .\@alph\c@subsubsubsection}
\newcommand\subsubsubsection{\@startsection{subsubsubsection}{4}{\z@}%
                                     {-3.25ex\@plus -1ex \@minus -.2ex}%
                                     {1.5ex \@plus .2ex}%
                                     {\normalfont\normalsize\bfseries}}
\renewcommand\paragraph{\@startsection{paragraph}{5}{\z@}%
                                    {3.25ex \@plus1ex \@minus.2ex}%
                                    {-1em}%
                                    {\normalfont\normalsize\bfseries}}
\renewcommand\subparagraph{\@startsection{subparagraph}{6}{\parindent}%
                                       {3.25ex \@plus1ex \@minus .2ex}%
                                       {-1em}%
                                      {\normalfont\normalsize\bfseries}}
\newcommand*\l@subsubsubsection{\@dottedtocline{4}{10.0em}{4.1em}}
\renewcommand*\l@paragraph{\@dottedtocline{5}{10em}{5em}}
\renewcommand*\l@subparagraph{\@dottedtocline{6}{12em}{6em}}
\newcommand*{\subsubsubsectionmark}[1]{}



\usepackage[english]{babel}															% English language/hyphenation
\usepackage[protrusion=true,expansion=true]{microtype}				% Better typography
\usepackage{amsmath,amsfonts,amsthm}										% Math packages
\usepackage[pdftex]{graphicx}														% Enable pdflatex
%\usepackage{color,transparent}													% If you use color and/or transparency
\usepackage[hang, small,labelfont=bf,up,textfont=it,up]{caption}	% Custom captions under/above floats
\usepackage{epstopdf}																	% Converts .eps to .pdf
\usepackage{subfig}																		% Subfigures
\usepackage{booktabs}																	% Nicer tables
\usepackage[utf8]{inputenc}

%%% Advanced verbatim environment
\usepackage{verbatim}
\usepackage{fancyvrb}
\DefineShortVerb{\|}								% delimiter to display inline verbatim text


%%% Custom sectioning (sectsty package)
\usepackage{sectsty}								% Custom sectioning (see below)
\allsectionsfont{%									% Change font of al section commands
	\usefont{OT1}{bch}{b}{n}%					% bch-b-n: CharterBT-Bold font
%	\hspace{15pt}%									% Uncomment for indentation
	}

\sectionfont{%										% Change font of \section command
	\usefont{OT1}{bch}{b}{n}%					% bch-b-n: CharterBT-Bold font
	\sectionrule{0pt}{0pt}{-5pt}{0.8pt}%	% Horizontal rule below section
	}


%%% Custom headers/footers (fancyhdr package)
\usepackage{fancyhdr}
\pagestyle{fancyplain}
\fancyhead{}														% No page header
\fancyfoot[C]{\thepage}										% Pagenumbering at center of footer
\fancyfoot[R]{\small \texttt{Master II 2014-2015}}	% You can remove/edit this line 
\renewcommand{\headrulewidth}{0pt}				% Remove header underlines
\renewcommand{\footrulewidth}{0pt}				% Remove footer underlines
\setlength{\headheight}{13.6pt}



%%% Equation and float numbering
\numberwithin{equation}{section}															% Equationnumbering: section.eq#
\numberwithin{figure}{section}																% Figurenumbering: section.fig#
\numberwithin{table}{section}																% Tablenumbering: section.tab#


%%% Title	
\title{ \vspace{-1in} 	\usefont{OT1}{bch}{b}{n}
		\huge \strut An overview of protein tertiary structure prediction and structurally informed function prediction \strut \\
		\Large \bfseries \strut Daniel Barry Roche \strut
}
\author{ 									\usefont{OT1}{bch}{m}{n}
        David Aubert \& Ahmed Rafik\\		\usefont{OT1}{bch}{m}{n}
        University of Montpellier\\	\usefont{OT1}{bch}{m}{n}
        Bio-informatics\\
        %\texttt{email@example.com}
}
\date{}


%%% Begin document
\begin{document}
\maketitle

\section{What is a protein structure prediction? }
\subsection{What is a protein tertiary structure? } 
\emph{The term \underline{protein tertiary structure} refers to a protein's geometric shape. The tertiary structure will have a single polypeptide chain "backbone" with one or more protein secondary structures, the protein domains. Amino acid side chains may interact and bond in a number of ways. The interactions and bonds of side chains within a particular protein determine its tertiary structure. The protein tertiary structure is defined by its atomic coordinates. These coordinates may refer either to a protein domain or to the entire tertiary structure.}
\newline \newline
The scientific vulgarization is that the tertiary structure is the spacial 3D structur. Her shape can change depending on his structure the pH and the temperature. A consequent number or properties can be found thank to the tertiary structure.
\newline
Thus, we can easily conclude why it is a major importance to modelize such data.


\subsection{Protein tertiary structure determination}

A protein tertiary structure determination allows us to know more on the protein we are currently observing such as:
\begin{itemize}
	\item Connection between sequence and structures
	\item  Easier than microscopic observation
	\item Evolution of proteins
\end{itemize}

\subsection{Different protein tertiary structure determination methods} 
\subsubsection{Template-based modelling}
\subsubsubsection{Template-based modelling fold recognition}
Similar protein's sequences have the same folds. Thus, we can classify proteins according to their shapes. One of the advantages of such classification is that the number of unique structural folds is very low compares to the number of proteins.
\newline
Today, the classification of folds is very advanced and only a few folds are discovered.
\subsubsubsection{Template-based modelling homology }

\subsubsection{Template-free modelling}
Template-free modeling is the prediction of the proteins structure. Compared to temple-based modelling, no proteins are used as template. 
\newline This technique has a lot of advantages compares to template-based ones.
\section{Critical Assessments  of Techniques  for Proteines Structural Prediction}
\subsection{What is it?}

It stands for a world championship of predictive structure. This "competition" has started in 1994, and it is define, by themselves their objective to be to help advance the methods of identifying protein structure sequence. They provide the architecture to make these research such as servers, samples and consulting.
\newline
With time the CASP became bigger and bigger. It can provides very advanced proteins modelisation tehcnics. A very big consortium has been created since then, with american research group such as Structural Genomics Consortium (SGC), New York Structural Genomics Research Center (NYSGRC).
\section{Model quality assessment}
\subsection{Model quality asessment algorithms}
\begin{enumerate}
	\item ModFOLDclust2
		\begin{enumerate}
		\item Clustering-based method
		\item Combines structural alignement of multiple models
		\item Producing glabal quality scores and per-residue errors
		\end{enumerate}
	\item RFMQA 
		\begin{enumerate}
		\item Single model-based method
		\item Random forest based model quality assessment
		\item Ranks protein models using its structural features and knowledge-based potential energy terms
		\item Produces global model quality score
		\end{enumerate}
\end{enumerate}

Basically, 

\subsection{Structurally informed function prediction}

There are many predictions ways existing for proteins structures. Most of them are structures observation methods.
\newline
\begin{enumerate}
\item Geometric methods
\item Energetic methods
\item Homology modelling
\item Surface accessibility
\item Physiochemical properties
\end{enumerate}
Also, others methods do exist. A sequence based method exist. It has significant impact on understanding protein function, elucidating signal transduction networks. This method accentuate the study of the amino acid sequence  and his prediction.
\newline 
This method is particuliary appreciated because the number of sequences to study is cosntantly growing and the sequence study take a consequent amount of time, the prediction will save time and will be able to reveal the protein's sequence.




\input{Structurally_informed_function_prediction}
\section{Protein ligand interaction prediction methods}
\subsection{FunFOLD}
The key features to understand FunFOLD is that despite the evolution (mutation, subsitution etc...) the structures, and by extention the folds, are more well-conserved than the sequence. Plus the ligand which allowed operation such as blocking a site, activate or deactivate protein's functions etc... are also well-conserved, this imply that the expression of the proteins depends on his structure and his ligands. Thus if a protein have the same ligand and structure we can imply that they have the same function.
\newline
\subsubsection{How does it work?}
The ligand could be considered as a tiny point of glue, it is first attached on a part of the protein, then it will carry the given information.
\newline
The prediction system work like this:
\newline
A ligand glue himself on a specific part of the protein and will leave there some marks. These residues can be visible but with some imprecision. The deposit can be realesed at a variable distance. This lack of accuracy can be compensate by the structure of the protein. Indeed, there are some structure that can promote a better accuracy. So the choice of the zone to study is very improtant. 
\newline
The FUNFold technic consist on the prediction capacity to know where the ligand are going to attached themselves and then to predict the function depending on the ligand expression.
\subsubsection{Benchmarking}
By comparing the predictions values returned by the FUNFold one compare to others technics used on the CASP, we can clearly see that the mean value is higher for the FUNFold. Thus FUNFold is more efficient.
\subsubsection{FUNFOLDQA: quality assement tool}
This quality assement for FUNFold is based on protein-ligand site residue. For starter, it is needed to have a 3D structures to analyse. Then the protein structure is analysed. Many methods are used:
\begin{enumerate}
\item{BDTalign: it's basically form recognition, the closest equivalent residues is choosen from a database.}
\item{Identify score: It's the same idea as BDTalign but instead of using the 3D form the amino acid sequence is searched and then the closest equivalent is choosen according to the amino acid sequence.}
\item{Rescaled BLOSUM62, Equivalent residue ligand distance score etc...}
\end{enumerate}

Basically most methods presented could be separated in two groups: using the structure of the protein and/or ligand or the composition of the protein/ligand.
\section{Limitations et perspectives}
\subsection{Limitation}
La première limitation est la dépendance au modèle 3D. Sans la modélisation de la protéïne il est impossible d'avancer. Cette dépendance est lié à la base de données utilisée, si la requête à la base de données est nulle, aucune prédiction sera faite.
\newline
Chaque séquence ne peut avoir qu'une seule prédiction, dans le cas d'une séquence avec plusieurs expressions nous serons vite limités.
\subsection{Perspectives}
La première amélioration importante serait l'augmentation de la précision pour la détection des résidus. Cette amélioration permettra directement d'augementer la précision dans la prédiction des protéines. Indirectement cette augmentation de précision passe par des contraintes logiciel tel que la modélisation 3D de la protéine. Pouvoir atténuer les distances entre les modèles  3D modèles et les protéines ciblées.
\newline
Ceci implique également une base de données consequente. En effet cet augmentation de précision va de pair avec volumes de données, plus la précision augmente plus les données pour les representer augmentent.





\iffalse 
\subsubsection{Heading on level 3 (subsubsection)}
Nulla consequat massa quis enim. Donec pede justo, fringilla vel, aliquet nec, vulputate eget, arcu. In enim justo, rhoncus ut, imperdiet a, venenatis vitae, justo. Nullam dictum felis eu pede mollis pretium. Integer tincidunt. Cras dapibus. Vivamus elementum semper nisi. Aenean vulputate eleifend tellus. Aenean leo ligula, porttitor eu, consequat vitae, eleifend ac, enim.

\paragraph{Heading on level 4 (paragraph)}
Lorem ipsum dolor sit amet, consectetuer adipiscing elit. Aenean commodo ligula eget dolor. Aenean massa. Cum sociis natoque penatibus et magnis dis parturient montes, nascetur ridiculus mus. Donec quam felis, ultricies nec, pellentesque eu, pretium quis, sem. Nulla consequat massa quis enim. 


\section{Lists}
\subsection{Example for list (itemize)}
\begin{itemize}
	\item First item in a list 
	\item Second item in a list 
	\item Third item in a list
\end{itemize}

\subsubsection{Example for list (3*itemize)}
\begin{itemize}
	\item First item in a list 
		\begin{itemize}
		\item First item in a list 
			\begin{itemize}
			\item First item in a list 
			\item Second item in a list 
			\end{itemize}
		\item Second item in a list 
		\end{itemize}
	\item Second item in a list 
\end{itemize}

\subsection{Example for list (enumerate)}
\begin{enumerate}
	\item First item in a list 
	\item Second item in a list 
	\item Third item in a list
\end{enumerate}

\subsubsection{Example for list (3*enumerate)}
\begin{enumerate}
	\item First item in a list 
		\begin{enumerate}
		\item First item in a list 
			\begin{enumerate}
			\item First item in a list 
			\item Second item in a list 
			\end{enumerate}
		\item Second item in a list 
		\end{enumerate}
	\item Second item in a list 
\end{enumerate}

\section{Mathematics}
Let's display some math:
\begin{align} 
	\begin{split}
	(x+y)^3 	&= (x+y)^2(x+y)\\
					&=(x^2+2xy+y^2)(x+y)\\
					&=(x^3+2x^2y+xy^2) + (x^2y+2xy^2+y^3)\\
					&=x^3+3x^2y+3xy^2+y^3
	\end{split}					
\end{align}

\begin{align}
	A = 
	\begin{bmatrix}
	A_{11} & A_{21} \\
  	A_{21} & A_{22}
	\end{bmatrix}
\end{align}
\fi

\end{document}
