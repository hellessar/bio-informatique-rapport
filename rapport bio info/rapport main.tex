%%% LaTeX Template
%%% This template can be used for both articles and reports.
%%%
%%% Copyright: http://www.howtotex.com/
%%% Date: February 2011

%%% Preamble
\documentclass[paper=a4, fontsize=11pt]{scrartcl}	% Article class of KOMA-script with 11pt font and a4 format

\setcounter{secnumdepth}{4}
\setcounter{tocdepth}{4}
\makeatletter
\newcounter {subsubsubsection}[subsubsection]
\renewcommand\thesubsubsubsection{\thesubsubsection .\@alph\c@subsubsubsection}
\newcommand\subsubsubsection{\@startsection{subsubsubsection}{4}{\z@}%
                                     {-3.25ex\@plus -1ex \@minus -.2ex}%
                                     {1.5ex \@plus .2ex}%
                                     {\normalfont\normalsize\bfseries}}
\renewcommand\paragraph{\@startsection{paragraph}{5}{\z@}%
                                    {3.25ex \@plus1ex \@minus.2ex}%
                                    {-1em}%
                                    {\normalfont\normalsize\bfseries}}
\renewcommand\subparagraph{\@startsection{subparagraph}{6}{\parindent}%
                                       {3.25ex \@plus1ex \@minus .2ex}%
                                       {-1em}%
                                      {\normalfont\normalsize\bfseries}}
\newcommand*\l@subsubsubsection{\@dottedtocline{4}{10.0em}{4.1em}}
\renewcommand*\l@paragraph{\@dottedtocline{5}{10em}{5em}}
\renewcommand*\l@subparagraph{\@dottedtocline{6}{12em}{6em}}
\newcommand*{\subsubsubsectionmark}[1]{}



\usepackage[english]{babel}															% English language/hyphenation
\usepackage[protrusion=true,expansion=true]{microtype}				% Better typography
\usepackage{amsmath,amsfonts,amsthm}										% Math packages
\usepackage[pdftex]{graphicx}														% Enable pdflatex
%\usepackage{color,transparent}													% If you use color and/or transparency
\usepackage[hang, small,labelfont=bf,up,textfont=it,up]{caption}	% Custom captions under/above floats
\usepackage{epstopdf}																	% Converts .eps to .pdf
\usepackage{subfig}																		% Subfigures
\usepackage{booktabs}																	% Nicer tables
\usepackage[utf8]{inputenc}

%%% Advanced verbatim environment
\usepackage{verbatim}
\usepackage{fancyvrb}
\DefineShortVerb{\|}								% delimiter to display inline verbatim text


%%% Custom sectioning (sectsty package)
\usepackage{sectsty}								% Custom sectioning (see below)
\allsectionsfont{%									% Change font of al section commands
	\usefont{OT1}{bch}{b}{n}%					% bch-b-n: CharterBT-Bold font
%	\hspace{15pt}%									% Uncomment for indentation
	}

\sectionfont{%										% Change font of \section command
	\usefont{OT1}{bch}{b}{n}%					% bch-b-n: CharterBT-Bold font
	\sectionrule{0pt}{0pt}{-5pt}{0.8pt}%	% Horizontal rule below section
	}


%%% Custom headers/footers (fancyhdr package)
\usepackage{fancyhdr}
\pagestyle{fancyplain}
\fancyhead{}														% No page header
\fancyfoot[C]{\thepage}										% Pagenumbering at center of footer
\fancyfoot[R]{\small \texttt{Master II 2014-2015}}	% You can remove/edit this line 
\renewcommand{\headrulewidth}{0pt}				% Remove header underlines
\renewcommand{\footrulewidth}{0pt}				% Remove footer underlines
\setlength{\headheight}{13.6pt}



%%% Equation and float numbering
\numberwithin{equation}{section}															% Equationnumbering: section.eq#
\numberwithin{figure}{section}																% Figurenumbering: section.fig#
\numberwithin{table}{section}																% Tablenumbering: section.tab#


%%% Title	
\title{ \vspace{-1in} 	\usefont{OT1}{bch}{b}{n}
		\huge \strut An overview of protein tertiary structure prediction and structurally informed function prediction \strut \\
		\Large \bfseries \strut Daniel Barry Roche \strut
}
\author{ 									\usefont{OT1}{bch}{m}{n}
        David Aubert \& Ahmed Rafik\\		\usefont{OT1}{bch}{m}{n}
        University of Montpellier\\	\usefont{OT1}{bch}{m}{n}
        Bio-informatics\\
        %\texttt{email@example.com}
}
\date{}


%%% Begin document
\begin{document}
\maketitle

\section{What is a protein structure prediction? }
\subsection{What is a protein tertiary structure? } 
\emph{The term \underline{protein tertiary structure} refers to a protein's geometric shape. The tertiary structure will have a single polypeptide chain "backbone" with one or more protein secondary structures, the protein domains. Amino acid side chains may interact and bond in a number of ways. The interactions and bonds of side chains within a particular protein determine its tertiary structure. The protein tertiary structure is defined by its atomic coordinates. These coordinates may refer either to a protein domain or to the entire tertiary structure.}
\newline \newline
The scientific vulgarization is that the tertiary structure is the spacial 3D structur. Her shape can change depending on his structure the pH and the temperature. A consequent number or properties can be found thank to the tertiary structure.
\newline
Thus, we can easily conclude why it is a major importance to modelize such data.

\subsection{Protein tertiary structure determination}
   
   Protein structure can be determinate experimentaly, using NMR or X-ray cristallography, and theoricaly, with template based method or template free method.\\

   \subsubsection{Why use this methods ?}

   The experimental method are very exhautive and hard to devellop. So the structure model determination are very important to predict tertiary structure.

   A protein tertiary structure determination allows us to know more on the protein we are currently observing such as:
      \begin{itemize}
	 \item Connection between sequence and structures
	 \item  Easier than microscopic observation
	 \item Evolution of proteins
      \end{itemize}
     
\subsection{Different protein tertiary structure determination methods} 
\subsubsection{Template-based modelling}
\subsubsubsection{Template-based modelling fold recognition}
Similar protein's sequences have the same folds. Thus, we can classify proteins according to their shapes. One of the advantages of such classification is that the number of unique structural folds is very low compares to the number of proteins.
\newline
Today, the classification of folds is very advanced and only a few folds are discovered.
\subsubsubsection{Template-based modelling homology }
Homology modelling is different and more complex than fold recognition. Instead of using the structural form of the protein the study of the amino acid sequence is used. Therefor, the complexity is bigger, the manippulation harder but the accuracy is more advance.

\subsubsection{Template-free modelling}
Template-free modeling is the prediction of the proteins structure. Compared to temple-based modelling, no proteins are used as template. 
\newline This technique has a lot of advantages compares to template-based ones. The protein will not be searched for his structural form or functions, but for some parts of his sequence. All proteins have commun function,  like energy functions or signals function for example. This technic is good because less database dependent than the templase-based one. Most functions are recorded, while functions can suffer mutation, transformation the function will not change or will be known.


\section{Critical Assessments  of Techniques  for Proteines Structural Prediction}
\subsection{What is it?}

It stands for a world championship of predictive structure. This "competition" has started in 1994, and it is define, by themselves their objective to be to help advance the methods of identifying protein structure sequence. They provide the architecture to make these research such as servers, samples and consulting.
\newline
With time the CASP became bigger and bigger. It can provides very advanced proteins modelisation tehcnics. A very big consortium has been created since then, with american research group such as Structural Genomics Consortium (SGC), New York Structural Genomics Research Center (NYSGRC).
\section{Model quality assessment}
\subsection{How to define the model quality}
   The model quality idea isn't easy to define because it's very subjective notion.\\
   It was define by comparison before the availability of cristal structure.\\

   Thus, with some evaluation criteria, Research could produce some algorithms which can evaluate models

\subsection{Evaluation Algorithm}
   \begin{enumerate}
	\item ModFOLDclust2
          Create by Daniel Roche and Mc Muffin in 2010, this algorithm operate as follows :
	  \begin{enumerate}
             \item Clustering-based method
             \item Combines structural alignement of multiple models with a method using Q-score
             \item Producing global quality scores and per-residue errors
	  \end{enumerate}
          So, this algorithm apply on a group of methods and compare their to know which is the best.
	\item RFMQA 
          This algorithm is much more recent, it was created en 2014 by Manavalan and Al
	  \begin{enumerate}
             \item It use a single model-based method
             \item Random forest based model quality assessment
             \item Ranks protein models using its structural features and knowledge-based potential energy terms
             \item Produces global model quality score
	  \end{enumerate}
          Also, this algorithm can use only one method to test it quality. it base on un random group of method referency and compare them to the first one to know if it's a good method.
          
   \end{enumerate}




\section{Structurally informed function prediction}
   \subsection{differents methods}
   
   There are two types of binding site prediction methods :
   \begin{enumerate}
     \item The sequence based method which identify conserved residues that may be structurally or functionally important
     \item The structure based method which is energetic, geometric method and use miscellaneous methods.
   \end{enumerate}

   \subsection{Function prediction in CASP}
      \subsubsection{What is it ?}
         
         That consist on make the prediction of ligand binding residues within a protein of unknown structure.
         
      \subsubsection{ligand binding site residue prediction methods}
         
         \begin{enumerate}
           \item Predict the location of the protein binding site
           \item Predict the ligand and location of the ligand within the binding site
           \item Predict the residues that bind to the ligand within the binding site
         \end{enumerate}

      \subsubsection{what's the utility ?}
         
         These tools are needed on many fields like annotation of genome, \textit{de novo} drug design, or mutagenesis studies.\\
         It's also used in the elucidation of protein function and to predict ligand binding specificity.

\section{Protein ligand interaction prediction methods}
\subsection{FUNFold}

\section{Limitations et perspectives}
\subsection{Limitation}
The first limitation is also 3D dependency. Without the protein's modelization it is impossible to go further. This dependancy is related to the database, if the query returns a null value, no prediction will be used.
\newline
Each sequency can only have one prediction, in that case a multi-expression sequency will be quickly eliminated and there will be a lack of accuracy in the prediction.
\subsection{Perspectives}
The first upgrade would be a higher ligand residue detection. This ugrade will allow to directly increase the protein's prediction. Indirectly, this accuracy rising will be related to the software constraint, such as the 3D recongnition. 
\newline Another upgrade could be a better 3D modelization that will allow a better flexibility in form recognition. This will allow to have better result between the model and the targeted protein.
\newline This implies a consequent architecture. Indeed, this accuracy increasement will be paired with data volume. The higher the  precision, the higher the data will be.





\iffalse 
\subsubsection{Heading on level 3 (subsubsection)}
Nulla consequat massa quis enim. Donec pede justo, fringilla vel, aliquet nec, vulputate eget, arcu. In enim justo, rhoncus ut, imperdiet a, venenatis vitae, justo. Nullam dictum felis eu pede mollis pretium. Integer tincidunt. Cras dapibus. Vivamus elementum semper nisi. Aenean vulputate eleifend tellus. Aenean leo ligula, porttitor eu, consequat vitae, eleifend ac, enim.

\paragraph{Heading on level 4 (paragraph)}
Lorem ipsum dolor sit amet, consectetuer adipiscing elit. Aenean commodo ligula eget dolor. Aenean massa. Cum sociis natoque penatibus et magnis dis parturient montes, nascetur ridiculus mus. Donec quam felis, ultricies nec, pellentesque eu, pretium quis, sem. Nulla consequat massa quis enim. 


\section{Lists}
\subsection{Example for list (itemize)}
\begin{itemize}
	\item First item in a list 
	\item Second item in a list 
	\item Third item in a list
\end{itemize}

\subsubsection{Example for list (3*itemize)}
\begin{itemize}
	\item First item in a list 
		\begin{itemize}
		\item First item in a list 
			\begin{itemize}
			\item First item in a list 
			\item Second item in a list 
			\end{itemize}
		\item Second item in a list 
		\end{itemize}
	\item Second item in a list 
\end{itemize}

\subsection{Example for list (enumerate)}
\begin{enumerate}
	\item First item in a list 
	\item Second item in a list 
	\item Third item in a list
\end{enumerate}

\subsubsection{Example for list (3*enumerate)}
\begin{enumerate}
	\item First item in a list 
		\begin{enumerate}
		\item First item in a list 
			\begin{enumerate}
			\item First item in a list 
			\item Second item in a list 
			\end{enumerate}
		\item Second item in a list 
		\end{enumerate}
	\item Second item in a list 
\end{enumerate}

\section{Mathematics}
Let's display some math:
\begin{align} 
	\begin{split}
	(x+y)^3 	&= (x+y)^2(x+y)\\
					&=(x^2+2xy+y^2)(x+y)\\
					&=(x^3+2x^2y+xy^2) + (x^2y+2xy^2+y^3)\\
					&=x^3+3x^2y+3xy^2+y^3
	\end{split}					
\end{align}

\begin{align}
	A = 
	\begin{bmatrix}
	A_{11} & A_{21} \\
  	A_{21} & A_{22}
	\end{bmatrix}
\end{align}
\fi

\end{document}
